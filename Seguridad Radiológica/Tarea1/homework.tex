\input{style/settings}
\input{style/short_commands}
\pagestyle{fancy}
\fancyhf{}
\fancyhead[R]{página\;\thepage/\pageref{LastPage}}
\fancyhead[L]{Osvaldo Uriel Calderón Dorantes}
\fancyfoot[L]{Seguridad Radiológica}
\fancyfoot[R]{Facultad de Ciencias, UNAM \includegraphics[scale=0.13]{style/Sheikah.pdf}}
\fancypagestyle{plain}{
  \fancyfoot[C]{}
}
\makeatletter
\def\@seccntformat#1{%
  \expandafter\ifx\csname c@#1\endcsname\c@section\else
  \csname the#1\endcsname\quad
  \fi}
\makeatother
%%%%%%%%%%%%%%%%%%%%%%%%%%%%%%%%%%%%%%%%%%%%%%%%%%%%%%%
%%%%%%%%%%%%%%%%%%%%%%%%%%%%%%%%%%%%%%%%%%%%%%%%%%%%%%%%%%%
\begin{document}
\begin{flushleft}
Osvaldo Uriel Calderón Dorantes, \hfill Seguridad Radiológica\\
316005171 \hfill osvaldo13576@ciencias.unam  \\
Facultad de Ciencias\\
\underline{Universidad Nacional Autónoma de México}
\end{flushleft}

\begin{flushright}\vspace{-5mm}
\includegraphics[height=1.5cm]{style/logo.pdf}
\end{flushright}
 
\begin{center}\vspace{-1cm}
\textbf{ \large \customfont{Tarea 1}}\\
\today
\end{center}
\medskip\hrule\medskip
%%%%%%%%%%%%%%%%%%%%%%%%%%%%%%%%%%%%%%%%%%%%%%%%
{\small \textbf{Nota: A las unidades las pondré dentro de corchetes \ec{\cor{\tx{\customfont{UNIDAD}}}} para no confundir entre variables y realizar el análisis dimensional fácilmente.}}
\medskip\hrule\bigskip

\newlength{\strutheight}
\settoheight{\strutheight}{\strut}


Recordemos que la ecuación que relaciona energía, masa y velocidad está dada como
\eq{e:emc2}{E=mc^2}
donde \ec{c=3\times10^{8}[m\cdot s^{-1}]} es la velocidad de la luz. Además la relación entre la unidad de energía 
\eq{e:j}{1[J]=1[kg\cdot m^2\cdot s^{-2}]}
y lo anterior se relaciona con el potencial eléctrico como
\eq{e:v}{1[V]=1[J\cdot C^{-1}]}
con la carga eléctrica y la unidad de masa atómica como 
\begin{align}
  e&=1.60218\times10^{-19}[C]\label{e:e}\\
  1[uma]&=1.66054\times10^{-27}[kg]\label{e:uma}
\end{align}


\begin{enumerate}[1.]
\item  Calcula la energía que le corresponde a un protón, cuya masa atómica es de \ec{1.6725\times 10^{-24}[g]}.

\section{Solución Ejercicio 1.}

Para este caso tenemos que \ec{m=1.6725\times 10^{-24}[g]=1.6725\times 10^{-27}[kg]}, por lo tanto la energía es
\al{E&=mc^2\\
     &=(1.6725\times 10^{-27}[kg])(3\times10^{8}[m\cdot s^{-1}])^2\\
     &=(1.6725\times 10^{-27})(9\times10^{16})[kg\cdot m^2\cdot s^{-2}]\\
     &=(1.6725\times 10^{-27})(9\times10^{16})[J]\times\p{\dfrac{e}{e}}\\
     &=(1.6725\times 10^{-27})(9\times10^{16})[J]\times\p{\dfrac{e}{1.60218\times10^{-19}[C]}}\\
     &=(1.6725\times 10^{-27})(9\times10^{10})\times\p{\dfrac{1}{1.60218\times10^{-19}}}[10^6\times e\cdot J\cdot C^{-1}]\\
     &=(1.6725\times 10^{-27})(9\times10^{10})\times\p{\dfrac{1}{1.60218\times10^{-19}}}[MeV]\\
     &=\dfrac{(1.6725)(9)\times10^{2}}{1.60218}[MeV]\\
     &=939.5012[MeV]
     }

\pagebreak

\item Calcula el equivalente de \ec{1[uma]} a \ec{MeV}.




\section{Solución Ejercicio 2.}


Para este caso tenemos que \ec{m=1[uma]=1.66054\times10^{-27}[kg]}, por lo tanto la energía es
\al{E&=mc^2\\
     &=(1.66054\times 10^{-27}[kg])(3\times10^{8}[m\cdot s^{-1}])^2\\
     &=(1.66054\times 10^{-27})(9\times10^{16})[kg\cdot m^2\cdot s^{-2}]\\
     &=(1.66054\times 10^{-27})(9\times10^{16})[J]\times\p{\dfrac{e}{e}}\\
     &=(1.66054\times 10^{-27})(9\times10^{16})[J]\times\p{\dfrac{e}{1.60218\times10^{-19}[C]}}\\
     &=(1.66054\times 10^{-27})(9\times10^{10})\times\p{\dfrac{1}{1.60218\times10^{-19}}}[10^6\times e\cdot J\cdot C^{-1}]\\
     &=(1.66054\times 10^{-27})(9\times10^{10})\times\p{\dfrac{1}{1.60218\times10^{-19}}}[MeV]\\
     &=\dfrac{(1.66054)(9)\times10^{2}}{1.60218}[MeV]\\
     &=932.78283[MeV]
     }


\item Calcula la energía en \ec{MeV} para un neutrón, cuya masa es de \ec{1.0086654[uma]}.

\section{Solución Ejercicio 3.}

Para este caso tenemos que \ec{m=1.0086654[uma]=1.0086654\times 1.66054\times10^{-27}[kg]}, por lo tanto la energía es
\al{E&=mc^2\\
     &=(1.0086654\times 1.66054\times 10^{-27}[kg])(3\times10^{8}[m\cdot s^{-1}])^2\\
     &=(1.0086654\times 1.66054\times 10^{-27})(9\times10^{16})[kg\cdot m^2\cdot s^{-2}]\\
     &=(1.0086654\times 1.66054\times 10^{-27})(9\times10^{16})[J]\times\p{\dfrac{e}{e}}\\
     &=(1.0086654\times 1.66054\times 10^{-27})(9\times10^{16})[J]\times\p{\dfrac{e}{1.60218\times10^{-19}[C]}}\\
     &=(1.0086654\times 1.66054\times 10^{-27})(9\times10^{10})\times\p{\dfrac{1}{1.60218\times10^{-19}}}[10^6\times e\cdot J\cdot C^{-1}]\\
     &=(1.0086654\times 1.66054\times 10^{-27})(9\times10^{10})\times\p{\dfrac{1}{1.60218\times10^{-19}}}[MeV]\\
     &=\dfrac{(1.0086654\times 1.66054)(9)\times10^{2}}{1.60218}[MeV]\\
     &=940.8657698[MeV]
     }

\end{enumerate}


%\begin{multicols}{2}
%\small{\bibliographystyle{apalike}
%\bibliography{bib}}
%\end{multicols}



%\ftikz{1.5}{figuras/fig.tikz}{}{fig:x}

\end{document}



}